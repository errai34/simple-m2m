\documentclass[12pt,preprint]{aastex}
\usepackage{amssymb,amsmath}
%\usepackage{color,hyperref}
% hypertex insanity
%\definecolor{linkcolor}{rgb}{0,0,0.25}
%\hypersetup{
%  colorlinks=true,        % false: boxed links; true: colored links
%  linkcolor=linkcolor,    % color of internal links
%  citecolor=linkcolor,    % color of links to bibliography
%  filecolor=linkcolor,    % color of file links
%  urlcolor=linkcolor      % color of external links
%}
\newcounter{address}
\setcounter{address}{1}
%\usepackage{sidecap}
\setlength{\emergencystretch}{2em}%No overflowing references
\newcommand{\ie}{i.e.}
\newcommand{\etal}{et al.}
\newcommand{\dd}{\mathrm{d}}
\newcommand{\eg}{e.g.}
\newcommand{\eqnname}{equation}
\newcommand{\equationname}{\eqnname}
\newcommand{\figurenames}{\figurename s}

\newcommand{\zsun}{\ensuremath{z_\odot}}
\newcommand{\vz}{\ensuremath{v_z}}
\newcommand{\vzi}{\ensuremath{v_{z,i}}}
\newcommand{\zi}{\ensuremath{z_i}}
\newcommand{\wi}{\ensuremath{w_i}}
\newcommand{\Ei}{\ensuremath{E_i}}
\newcommand{\Ai}{\ensuremath{A_i}}
\newcommand{\phii}{\ensuremath{\phi_i}}
\newcommand{\zobs}{\ensuremath{\tilde{z}}}
\newcommand{\eps}{\ensuremath{\epsilon}}

\begin{document}

\title{Harmonic-oscillator M2M}
\author{Jo~Bovy\altaffilmark{1,2,3}}
\altaffiltext{\theaddress}{\label{1}\stepcounter{address}
  Department of Astronomy and Astrophysics, University of Toronto, 50
  St.  George Street, Toronto, ON, M5S 3H4, Canada;
  bovy@astro.utoronto.ca~}
\altaffiltext{\theaddress}{\label{2}\stepcounter{address} 
  Center for Computational Astrophysics, Flatiron Institute, 162 5th Ave, New York, NY 10010, USA}
\altaffiltext{\theaddress}{\label{3}\stepcounter{address}
  Alfred~P.~Sloan~Fellow}

\begin{abstract} 
  This note presents some enhancements to the standard
  \emph{made-to-measure} (M2M) approach for fitting steady-state
  distribution functions to kinematic data in the context of a simple
  one-dimensional harmonic-oscillator potential. Specifically, this
  note aims to do the following:\\
  $\bullet$ Fit for nuisance parameters in addition to the orbital
  weights; specifically, fit for the Sun's height \zsun\ above the
  midplane;\\ $\bullet$ Fit for the parameters of the potential;
  specifically, fit for the period $\omega$ of the harmonic
  oscillator;\\ $\bullet$ Run MCMC for the weights using Hybrid Monte
  Carlo;\\ $\bullet$ Run MCMC for the nuisance parameters using Hybrid
  Monte Carlo as well;\\ $\bullet$ Run MCMC for the potential
  parameters.\\
  With these improvements, M2M can be used to fully fit observable
  data and return proper uncertainties on all parameters.
\end{abstract}

\section{Online material}

These notes accompany an online ipython notebook that contains
experiments and figures using the methodology below. This notebook is
located at
\\ \centerline{\texttt{\url{https://github.com/jobovy/simple-m2m/blob/master/py/HOM2M.ipynb}}}
The few movies that are part of this notebook can only be viewed on nbviewer at
\\ \centerline{\footnotesize \texttt{\url{http://nbviewer.jupyter.org/github/jobovy/simple-m2m/blob/master/py/HOM2M.ipynb}}}.

\section{Introduction}

For now, the reader is assumed to be familiar with the standard
made-to-measure modeling technique. If not, \citet{Syer96a} and
\citet{Dehnen09a} are useful references to get started.

\section{Setup}

We consider a simple one-dimensional system where the potential is a
simple harmonic oscillator. The phase-space coordinates are $(z,\vz)$
and the potential is
\begin{equation}
  \Phi(z) = \frac{\omega^2\,z^2}{2}\,.
\end{equation}
In this potential, we attempt to match a population drawn from the
following distribution function
\begin{equation}
  f(z,\vz) \propto e^{-E/\sigma_t^2}\,,
\end{equation}
where $E = \omega^2\,z^2 / 2 + \vz^2/2$ is the energy and $\sigma_t =
0.1$ is the true velocity dispersion. This distribution function is
isothermal, it has the same velocity dispersion at all heights.

To fit this distribution function using M2M, we start with
$(\zi,\vzi)$ drawn with uniform weights \wi\ from a similar isothermal
distribution function, but with a larger $\sigma$: $f(z,\vz) \propto
e^{-E/\sigma_{\mathrm{in}}^2}$, with $\sigma_{\mathrm{in}}$ typically
$0.2$. It is then easy to see that the correct output weights should
be
\begin{equation}
  \zi = \exp\left( -\Ei\left[1/\sigma_t^2-1/\sigma_{\mathrm{in}}^2\right]\right)\,.
\end{equation}

Orbit integration in the harmonic-oscillator potential is analytic and
we simply have that
\begin{align}\label{eq:zit}
  \zi(t) & = \phantom{-}\Ai\phantom{\,\omega}\,\cos\left(\omega\,t+\phii\right)\,,\\
  \vzi(t) & = -\Ai\,\omega\,\sin\left(\omega\,t+\phii\right)\,,\label{eq:vzit}
\end{align}
where
\begin{align}\label{eq:Ai}
  A_i  & = \frac{\sqrt{2\,E_i}}{\omega} = \sqrt{z_i^2(0)+\vzi^2(0)/\omega^2}\,,\\
  \phii & = \arctan\!2(-\vzi(0)/\omega,\zi(0))\,,\label{eq:phii}
\end{align}
in which $(\zi(0),\vzi(0))$ is the initial phase-space position and
$\arctan\!2$ is the arc-tangent function that chooses the quadrant
correctly. Sampling initial $(\zi(0),\vzi(0))$ from $f(z,\vz) \propto
e^{-E/\sigma^2}$ is simple: sample $\Ei$ from the exponential
distribution and convert it to $\Ai$; sample $\phii$ uniformly between
$0$ and $2\pi$.

\section{Standard M2M}

We first describe the standard M2M case. Standard M2M models a
steady-state distribution function as a set of $N$ particles
($\zi,\vzi)$ indexed by $i$ orbiting in a fixed potential. Each
particle has a weight $\wi$ that is adjusted on-the-fly during orbit
integration to fit a set of constraints, like the density in bins, or
the velocity dispersion. By only adjusting the weights $\wi$ on
timescales $\gg$ the orbital timescale, an approximate equilibrium
distribution is obtained\footnote{Unlike in Schwarzschild modeling,
  where orbits are integrated for hundreds of dynamical times and the
  observables are fit using these orbits, this is a more approximate
  equilibrium. But in some sense it is \emph{exactly} the correct
  state to fit, because stellar populations are only in a
  quasi-stationary state for a few orbital periods (on longer
  timescales they will evolve due to interactions with clouds, spiral
  arms, satellite galaxies, etc. or due to the evolution of the
  underlying gravitational potential.}.

In practice, M2M maximizes an objective function $F$ that represents a
balance between reproducing the constraints and smoothness of the
distribution function (through a maximum-entropy constraint)
\begin{equation}
  F = \mu S - \frac{1}{2}\sum_j \chi^2_j\,
\end{equation}
where $S = - \sum_i w_i \ln\left(w_i/\hat{w}_i\right)$ is the entropy
and $\chi_j^2$ are constraints expressed in a $\chi^2$ type
manner. Constraints are expressed as a kernel applied to the
distribution function $f(z,\vz)$:
\begin{equation}
  Y_j = \int \dd z\dd \vz \,K_j(z,\vz) f(z,\vz)\,
\end{equation}
which for the $N$-body relation is computed as
\begin{equation}
  y_j = \sum_i \wi K_j(\zi,\vzi)\,.
\end{equation}

To illustrate the standard M2M case, we use the density observed at a
few points, so
\begin{equation}
  \rho(\zobs_j) = \frac{1}{N}\sum_i \wi K^\rho(|\zobs_j+\zsun - \zi|;h)\,,
\end{equation}
where $K^\rho(r;h)$ is a kernel function that integrates to one ($\int
\dd r\,K^\rho(r;h) = 1$), we have assumed that $\sum_i \wi = N$, and
we assume that the observations are done as a function of $\zobs$,
which is measured with respect to the Sun's position, located at \zsun\
from the $z=0$ midplane. We form $\chi^2_j$ as
\begin{equation}
  \chi_j^2 = [\Delta^\rho_j]^2 = \left( \rho(\zobs_j)/\rho^\mathrm{obs}_j-1\right)^2\,.
\end{equation}

The M2M \emph{force of change} equation is then given by
\begin{equation}
\begin{split}\label{eq:fcw}
  \frac{\dd \wi}{\dd t} & = \eps \wi\,\frac{\partial F}{\partial \wi}\\
  & = -\eps \wi \,\left[\mu \left(\ln\left[w_i/\hat{w}_i\right]+1\right)+ \sum_j \Delta^\rho_j\, K^\rho_j(\zi;h)/\rho_j^\mathrm{obs} \right]\,.
\end{split}
\end{equation}
We solve this equation using a simple Euler method with a fixed step
size, computing the orbital evolution as we go along using
\equationname s~(\ref{eq:zit}) and (\ref{eq:vzit}). This method for
optimizing the objective function can be thought of as a sort of
gradient ascent. In this interpretation, \eps\ is adjusted such that
substantial changes to the orbital weights only happen on timescales
$\gg$ the orbital timescale, which pushes the weights to an
equilibrium distribution.

\citet{Syer96a} propose to lessen the impact of Poisson noise due to
the finite number of $N$-body particles by smoothing the
$\Delta^\rho_j$ deviation that appears in \equationname~(\ref{eq:fcw})
with a smoothed version $\tilde{\Delta}_j$. In the end, this leads one
to solve for $\tilde{\Delta}_j$ using the differential equation
\begin{equation}
  \frac{\dd \tilde{\Delta}_j}{\dd t} = \alpha \left(\Delta_j-\tilde{\Delta}_j\right)\,,
\end{equation}
where $\alpha$ is another timescale parameter. Because we only want to
smooth on shorter timescales than that over which we substantially
change the weights, we typically need $\alpha > \eps$.

\section{Fitting for the nuisance parameter \protect\zsun}

In the observed density above, we have assumed that we know the Sun's
distance from the midplane from which the stars are observed. Now
suppose that we do not know \zsun. Then when we compare the density to
the observed density $\rho(\zobs)$ as a function of \zobs, we do not
know how to shift the model \zi\ to \zobs. The standard approach would
be to run multiple instances of the standard M2M algorithm for
different values of \zsun\ and to then determine the maximum of
$F(\zsun)$. This becomes expensive if there are many nuisance
parameters.

Here we propose to simply update \zsun\ in tandem with the weights. To
do this, we compute the force of change for \zsun, which in the case
of fitting the density is
\begin{equation}
\frac{\dd \zsun}{\dd t} = -\eps_z\,\sum_j \Delta^\rho_j/\rho^{\mathrm{obs}}_j \sum_i w_i\,\frac{\dd K^\rho(r)}{\dd r}\Bigg|_{|\zobs+\zsun-\zi|}\,\mathrm{sign}(\zobs+\zsun-\zi)\,,
\end{equation}
where we have introduced a different $\eps_z$ parameter to control how
fast $\zsun$ is changed during the maximization. We expect that we
want $\eps_z < \eps$, because we want to change \zsun\ only on
timescales that are longer than the timescales over which we adjust
the weights.

\section{Fitting for the potential parameter $\protect\omega$}

Next, we want to fit for the potential parameter $\omega$ as well. To
do this, we need to introduce a velocity constraint as well, because
the potential cannot be constrained using the density alone. For
simplicity, we choose the velocity constraint to be the density times
the mean squared velocity
\begin{equation}
\begin{split}
  \rho\times\langle v^2 \rangle (\zobs_j) & = \frac{1}{N}\sum_i \wi K^v(|\zobs_j+\zsun - \zi|,\vzi;h)\\
  & = \frac{1}{N}\sum_i \wi\,vzi^2\, K^\rho(|\zobs_j+\zsun - \zi|;h)\,,
\end{split}
\end{equation}
where we have set $K^v(z,\vz) = v^2\,K^\rho(z)$. We assume that each
velocity constraint adds the following to the total $\chi^2$
\begin{equation}
\chi^2_{j,v} = [\Delta^v_j]^2 = \left(\frac{\rho\times\langle v^2 \rangle (\zobs_j)}{[\rho\times\langle v^2 \rangle]_j^{\mathrm{obs}}}-1\right)\,.
\end{equation}

The force of change equation for the weights then gets an additional
contribution due to the velocity constraint. This additional
contribution (which gets added to the right-hand side of
\equationname~[\ref{eq:fcw}]) is
\begin{equation}\label{eq:fcwv2}
    -\eps_v \wi \,\sum_j \Delta^v_j\, \vzi^2\,K^\rho_j(|\zobs_j+\zsun - \zi|;h)/[\rho\times\langle v^2 \rangle]_j^{\mathrm{obs}}\,.
\end{equation}
Here we have kept the freedom to use a different epsilon parameter
$\eps_v$ to adjust the weighting between the density and velocity
constraints.

Similar to the force of change for \zsun, we can also compute the
force of change for $\omega$. This is given by
\begin{equation}\label{eq:fcomega}
\begin{split}
  \frac{\dd \omega}{\dd t} = & -\eps_\omega\,\sum_j \sum_i \wi \Bigg[\left\{\frac{\Delta^\rho_j}{\rho^{\mathrm{obs}}_j}+\frac{\Delta^v_j}{[\rho\times\langle v^2 \rangle]_j^{\mathrm{obs}}}\,\vzi^2\right\}\,\frac{\dd K^\rho(r)}{\dd r}\Bigg|_{|\zobs+\zsun-\zi|}\,\mathrm{sign}(\zobs-\zsun-\zi)\frac{\partial \zi}{\partial \omega}\,\\
  & \qquad \qquad \qquad \qquad +2\frac{\Delta^v_j}{[\rho\times\langle v^2 \rangle]_j^{\mathrm{obs}}} \,\vzi\,K^\rho(|\zobs+\zsun-\zi|)\,\frac{\partial \vzi}{\partial \omega}\Bigg]\,.
\end{split}
\end{equation}
The problem is now that the instantaneous phase-space position
$(\zi,\vzi)$ does not depend on $\omega$.

Nevertheless, we can compute the partial derivatives $\partial \zi /
\partial \omega$ and $\partial \vzi / \partial \omega$ using
\equationname s~(\ref{eq:zit}) and (\ref{eq:vzit}). This gives
\begin{align}
  \frac{\partial \zi}{\partial \omega} & = \frac{t}{\omega} \vzi\,,\\
  \frac{\partial \vzi}{\partial \omega} & = -\omega\,t \zi\,.
\end{align}
At $t=0$ these derivatives are zero (because the instantaneous
phase-space position does not depend on the potential), but if we
integrate for a short time, the phase-space positions computed in
different potentials would be different, as given by these partial
derivatives for short times. Thus, we can drop these into
\equationname~(\ref{eq:fcomega}) and absorb the $t$ parameter in the
above equation in $\eps_\omega$. We thus change the potential
parameter $\omega$ on-the-fly as well, again requiring $\eps_\omega$
to be such that substantial changes to $\omega$ only occur on
timescales $\gg$ than the timescale over which substantial changes in
the weights occur. Because we change $\omega$ on the fly, each time
step, we need to re-compute the $(\Ai,\phii)$ parameters using
\equationname s~(\ref{eq:Ai}) and (\ref{eq:phii}).

This appears to work quite well! For more complicated potentials,
these partial derivatives can be computed by solving the differential
equation for the difference in $z$ and $\vz$ for small potential
changes.



\begin{thebibliography}{}
%\bibitem[Bovy(2015)]{Bovy15a}
%  Bovy, J. 2015, \apjs, 216, 29
\bibitem[Syer \& Tremaine(1996)]{Syer96a} Syer, D., \& Tremaine,
  S.\ 1996, \mnras, 282, 223
\bibitem[Dehnen(2009)]{Dehnen09a}
  Dehnen, W.\ 2009, \mnras, 395, 1079
\end{thebibliography}

\end{document}

